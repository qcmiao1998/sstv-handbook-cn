\chapter*{慢扫描电视}
\addcontentsline{toc}{chapter}{慢扫描电视}
\markboth{慢扫描电视}{}

慢扫描电视(SSTV)是用于传输图像的一种通信模式。慢扫描电视是一种窄带通信,它可以用从单边带(SSB)电台的音频通道发射到业余频段上。在传播良好的情况下,高频段的世界范围的通信也是可能的。

\section{起源}
在1957年,一名肯塔基大学的学生Copthorne “Cop” Macdonalds WA2BCW (现VY2CM),发现了一篇论文,讲述了用贝尔实验室发明的设备在电话线上传输图像。这个通信系统让这位业余无线电爱好者着迷,因为它需要的带宽和话音一样窄,而且能够用一般的业余无线电台传输。\\

在当时另一种图像模式无线电传真(Facsimile)是可用的,但是它需要更长的时间(约20分钟)来传送高分辨率的图像。这样的持续时长不能在一次QSO中提供时间一致性,并且需要一台复杂的机械打印机和电敏纸。有必要发明另一种方式。\\

有一种想法是将要传输的图像编码到音频信号中,并显示在长余辉显示器(用于雷达或慢扫描示波器的显像管)上。\\

然后Copthorne开始研究怎样将图像通过业余收发信机在无线电波传输。他在六个月内做了许多关于调频和调谐的实验,慢扫描电视的设计就从中诞生。接下来的六个月他制造了一台SSTV图像扫描仪,这样实验就能在业余频段开展了。在1959年12月20日,第一张电视图像跨越了大西洋。\\


% 第一张跨越大西洋的图像,由John Plowman G3AST接收
% Copthorne Macdonald’s正在播报
% 早期SSTV图像



在接下来的十年里Copthorne和一群业余爱好者继续改进SSTV,他们建立了基本的SSTV标准并开发了一种取样相机。他们的工作于1968年完成。美国联邦通信委员会(FCC)正式授权了SSTV操作。\\

几个月后业余无线电杂志刊登了第一篇关于这种新通信模式的文章。这篇文章引发了火腿们极大的兴趣和一个真正的SSTV热潮。\\

\section{图像传输}

SSTV的基本思想是使用标准收发信机传输电视图像。然而,电视广播需要更宽的频宽,电视信号的缩减需要靠降低水平(行)和垂直(帧)扫描的次数来实现,且必须缩减到最少的频率。这就意味着一个典型的黑白电视信号必须从3MHz频宽缩减到3kHz——缩减率接近1000:1。如今频宽缩减更加严重,因为彩色电视图像需要将近6MHz的频宽。由于巨大的频宽缩减,只有低分辨率的静止图像才能被传输。\\

在实验中,我们发现图像可以在带有P7磷光体的长余辉显像管上保持可见约8秒。所以在接收完最后一行后,第一行依然可见,但在一段时间内会慢慢消失。为了达到最佳效果,需要在黑暗的房间里观看SSTV显示器。通常几个相同的图像会一次传输,每个后续图像会缓慢地将原始图像还原在磷光体上,原始图像依然可见。这样就允许图像显示更长的时间或录像以供之后回放。\\

我们发现电子电路正确检测出行同步脉冲的理想时间是5毫秒,检测出帧(垂直)同步脉冲的理想时间是30毫秒。帧同步开启显像管图像显示的自动启动。\\

扫描线和帧的同步频率是从电源频率推导出的。水平扫描的频率是50Hz/3=16.6Hz,垂直扫描的频率是1/7.2s=0.1388Hz,这是由电源频率除以360(3×120行)得出的。在电源频率是60Hz的国家,这些频率的推导方法相同。\\

视频信号的频带的选择范围在1500Hz为黑色,2300Hz为白色之间。同步脉冲在1200Hz频率上,因为它比“黑色信号还要黑”,所以不会影响图像信息。所有SSTV频率分量都在低频带内,可以通过语音频道传输。\\

从这个原始标准产生的其他SSTV模式,它们绝大部分都只在扫描速度和彩色传输的增加方面不同。\\